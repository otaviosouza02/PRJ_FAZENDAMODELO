% Options for packages loaded elsewhere
\PassOptionsToPackage{unicode}{hyperref}
\PassOptionsToPackage{hyphens}{url}
%
\documentclass[
]{article}
\usepackage{amsmath,amssymb}
\usepackage{iftex}
\ifPDFTeX
  \usepackage[T1]{fontenc}
  \usepackage[utf8]{inputenc}
  \usepackage{textcomp} % provide euro and other symbols
\else % if luatex or xetex
  \usepackage{unicode-math} % this also loads fontspec
  \defaultfontfeatures{Scale=MatchLowercase}
  \defaultfontfeatures[\rmfamily]{Ligatures=TeX,Scale=1}
\fi
\usepackage{lmodern}
\ifPDFTeX\else
  % xetex/luatex font selection
\fi
% Use upquote if available, for straight quotes in verbatim environments
\IfFileExists{upquote.sty}{\usepackage{upquote}}{}
\IfFileExists{microtype.sty}{% use microtype if available
  \usepackage[]{microtype}
  \UseMicrotypeSet[protrusion]{basicmath} % disable protrusion for tt fonts
}{}
\makeatletter
\@ifundefined{KOMAClassName}{% if non-KOMA class
  \IfFileExists{parskip.sty}{%
    \usepackage{parskip}
  }{% else
    \setlength{\parindent}{0pt}
    \setlength{\parskip}{6pt plus 2pt minus 1pt}}
}{% if KOMA class
  \KOMAoptions{parskip=half}}
\makeatother
\usepackage{xcolor}
\usepackage[margin=1in]{geometry}
\usepackage{color}
\usepackage{fancyvrb}
\newcommand{\VerbBar}{|}
\newcommand{\VERB}{\Verb[commandchars=\\\{\}]}
\DefineVerbatimEnvironment{Highlighting}{Verbatim}{commandchars=\\\{\}}
% Add ',fontsize=\small' for more characters per line
\usepackage{framed}
\definecolor{shadecolor}{RGB}{248,248,248}
\newenvironment{Shaded}{\begin{snugshade}}{\end{snugshade}}
\newcommand{\AlertTok}[1]{\textcolor[rgb]{0.94,0.16,0.16}{#1}}
\newcommand{\AnnotationTok}[1]{\textcolor[rgb]{0.56,0.35,0.01}{\textbf{\textit{#1}}}}
\newcommand{\AttributeTok}[1]{\textcolor[rgb]{0.13,0.29,0.53}{#1}}
\newcommand{\BaseNTok}[1]{\textcolor[rgb]{0.00,0.00,0.81}{#1}}
\newcommand{\BuiltInTok}[1]{#1}
\newcommand{\CharTok}[1]{\textcolor[rgb]{0.31,0.60,0.02}{#1}}
\newcommand{\CommentTok}[1]{\textcolor[rgb]{0.56,0.35,0.01}{\textit{#1}}}
\newcommand{\CommentVarTok}[1]{\textcolor[rgb]{0.56,0.35,0.01}{\textbf{\textit{#1}}}}
\newcommand{\ConstantTok}[1]{\textcolor[rgb]{0.56,0.35,0.01}{#1}}
\newcommand{\ControlFlowTok}[1]{\textcolor[rgb]{0.13,0.29,0.53}{\textbf{#1}}}
\newcommand{\DataTypeTok}[1]{\textcolor[rgb]{0.13,0.29,0.53}{#1}}
\newcommand{\DecValTok}[1]{\textcolor[rgb]{0.00,0.00,0.81}{#1}}
\newcommand{\DocumentationTok}[1]{\textcolor[rgb]{0.56,0.35,0.01}{\textbf{\textit{#1}}}}
\newcommand{\ErrorTok}[1]{\textcolor[rgb]{0.64,0.00,0.00}{\textbf{#1}}}
\newcommand{\ExtensionTok}[1]{#1}
\newcommand{\FloatTok}[1]{\textcolor[rgb]{0.00,0.00,0.81}{#1}}
\newcommand{\FunctionTok}[1]{\textcolor[rgb]{0.13,0.29,0.53}{\textbf{#1}}}
\newcommand{\ImportTok}[1]{#1}
\newcommand{\InformationTok}[1]{\textcolor[rgb]{0.56,0.35,0.01}{\textbf{\textit{#1}}}}
\newcommand{\KeywordTok}[1]{\textcolor[rgb]{0.13,0.29,0.53}{\textbf{#1}}}
\newcommand{\NormalTok}[1]{#1}
\newcommand{\OperatorTok}[1]{\textcolor[rgb]{0.81,0.36,0.00}{\textbf{#1}}}
\newcommand{\OtherTok}[1]{\textcolor[rgb]{0.56,0.35,0.01}{#1}}
\newcommand{\PreprocessorTok}[1]{\textcolor[rgb]{0.56,0.35,0.01}{\textit{#1}}}
\newcommand{\RegionMarkerTok}[1]{#1}
\newcommand{\SpecialCharTok}[1]{\textcolor[rgb]{0.81,0.36,0.00}{\textbf{#1}}}
\newcommand{\SpecialStringTok}[1]{\textcolor[rgb]{0.31,0.60,0.02}{#1}}
\newcommand{\StringTok}[1]{\textcolor[rgb]{0.31,0.60,0.02}{#1}}
\newcommand{\VariableTok}[1]{\textcolor[rgb]{0.00,0.00,0.00}{#1}}
\newcommand{\VerbatimStringTok}[1]{\textcolor[rgb]{0.31,0.60,0.02}{#1}}
\newcommand{\WarningTok}[1]{\textcolor[rgb]{0.56,0.35,0.01}{\textbf{\textit{#1}}}}
\usepackage{graphicx}
\makeatletter
\def\maxwidth{\ifdim\Gin@nat@width>\linewidth\linewidth\else\Gin@nat@width\fi}
\def\maxheight{\ifdim\Gin@nat@height>\textheight\textheight\else\Gin@nat@height\fi}
\makeatother
% Scale images if necessary, so that they will not overflow the page
% margins by default, and it is still possible to overwrite the defaults
% using explicit options in \includegraphics[width, height, ...]{}
\setkeys{Gin}{width=\maxwidth,height=\maxheight,keepaspectratio}
% Set default figure placement to htbp
\makeatletter
\def\fps@figure{htbp}
\makeatother
\setlength{\emergencystretch}{3em} % prevent overfull lines
\providecommand{\tightlist}{%
  \setlength{\itemsep}{0pt}\setlength{\parskip}{0pt}}
\setcounter{secnumdepth}{5}
\ifLuaTeX
  \usepackage{selnolig}  % disable illegal ligatures
\fi
\usepackage[]{natbib}
\bibliographystyle{plainnat}
\usepackage{bookmark}
\IfFileExists{xurl.sty}{\usepackage{xurl}}{} % add URL line breaks if available
\urlstyle{same}
\hypersetup{
  pdftitle={Descrição das etapas de processamento de dados de inventário realizado com o auxílio da tecnologia LiDAR},
  pdfauthor={Otávio Magalhães Silva Souza},
  hidelinks,
  pdfcreator={LaTeX via pandoc}}

\title{Descrição das etapas de processamento de dados de inventário
realizado com o auxílio da tecnologia LiDAR}
\author{Otávio Magalhães Silva Souza}
\date{}

\begin{document}
\maketitle

\begin{center}\includegraphics[width=0.4\linewidth]{IMAGES/CAPA} \end{center}

\centerline {Piracicaba, SP – Data de Emissão: 13 de agosto de 2024}
\newpage

\tableofcontents

\newpage

\section{Pacotes utilizados no R (colocar breve descrição - já tem uma
descriçãozinha no R passado em
aula)}\label{pacotes-utilizados-no-r-colocar-breve-descriuxe7uxe3o---juxe1-tem-uma-descriuxe7uxe3ozinha-no-r-passado-em-aula}

\subsection{\texorpdfstring{Tidyverse
(\url{https://livro.curso-r.com/4-2-tidyverse.html})}{Tidyverse (https://livro.curso-r.com/4-2-tidyverse.html)}}\label{tidyverse-httpslivro.curso-r.com4-2-tidyverse.html}

O Tidyverse é um pacote guarda-chuva e contém diversas funções úteis
para garantir o dinamismo no script, visualização, processamento e
análise dos dados, modelagem etc.

\begin{figure}

{\centering \includegraphics[width=0.6\linewidth]{IMAGES/tidyverse} 

}

\caption{Tidyverse}\label{fig:unnamed-chunk-2}
\end{figure}

\subsection{Sf}\label{sf}

Pacote utilizado para manipulação de objetos do mundo real. Descreve a
forma com que esses objetos podem ser armazenados e importados e quais
operações geométricas podem ser definidas por eles.

\subsection{Tidyterra}\label{tidyterra}

\subsection{Terra}\label{terra}

\subsection{Stars}\label{stars}

\subsection{Tools}\label{tools}

\subsection{RColorBrewer}\label{rcolorbrewer}

\subsection{Progress}\label{progress}

\subsection{Reshape2}\label{reshape2}

\subsection{Mapview}\label{mapview}

\subsection{LidR}\label{lidr}

\subsection{RCSF}\label{rcsf}

\subsection{Future}\label{future}

\newpage

\section{Descrição da área}\label{descriuxe7uxe3o-da-uxe1rea}

A área a ser estudada como ``Fazenda Modelo'' localiza-se no município
de São Miguel Arcanjo (SP), pode ser identificada pelas coordenadas
(-23.86707°, -47.87772°) e possui 129,784 ha, que dividem-se em 4
subtalhões: 301a (18,933 ha), 301d (34,468 ha), 302a (47,602 ha) e 302c
(28,781 ha).

\begin{figure}

{\centering \includegraphics[width=0.6\linewidth]{IMAGES/mapafazendamodelo} 

}

\caption{Mapa da propriedade}\label{fig:unnamed-chunk-3}
\end{figure}
\begin{figure}

{\centering \includegraphics[width=0.4\linewidth]{IMAGES/nuvensnormalizadas} \includegraphics[width=0.4\linewidth]{IMAGES/nuvensnormalizadaszoom} 

}

\caption{Nuvens LiDAR normalizadas}\label{fig:unnamed-chunk-4}
\end{figure}

\newpage

\section{Grid e parcelas já
inventariadas}\label{grid-e-parcelas-juxe1-inventariadas}

A região foi dividida em 3454 parcelas, onde 2960 delas possuem 400m²,
enquanto as outras são menores por estarem na borda e abrangerem áreas
além da área de interesse. Além disso, 13 das parcelas possuem dados de
inventário florestal e podem ser identificadas pelos seguintes Id's:
993, 1526, 1770, 1881, 3165, 3628, 3660, 3730, 5052, 5091, 5106 e 5122.

\begin{figure}

{\centering \includegraphics[width=0.4\linewidth]{IMAGES/parcelasinventariadas} 

}

\caption{Parcelas com dados de inventário}\label{fig:unnamed-chunk-5}
\end{figure}

\newpage

\section{Processamento da nuvem
LiDAR}\label{processamento-da-nuvem-lidar}

\begin{figure}

{\centering \includegraphics[width=0.8\linewidth]{IMAGES/fluxograma-funcoes} 

}

\caption{Fluxograma do processamento da nuvem LiDAR}\label{fig:unnamed-chunk-6}
\end{figure}

\begin{itemize}
\tightlist
\item
  \begin{enumerate}
  \def\labelenumi{\arabic{enumi}.}
  \tightlist
  \item
    Conferir especificações: Verificar se a nuvem gerada se enquadra no
    que é esperado: densidade de pontos, área escaneada, classificação
    de pontos etc;
  \end{enumerate}
\item
  \begin{enumerate}
  \def\labelenumi{\arabic{enumi}.}
  \setcounter{enumi}{1}
  \tightlist
  \item
    Arrumar nuvens: Verificação do sistema de coordenadas utilizado e
    atributos de latitude, longitude e elevação;
  \end{enumerate}
\item
  \begin{enumerate}
  \def\labelenumi{\arabic{enumi}.}
  \setcounter{enumi}{2}
  \tightlist
  \item
    Quadricular: Repartição da nuvem em pedaços menores a fim de ganhar
    eficiência no processamento em paralelo;
  \end{enumerate}
\item
  \begin{enumerate}
  \def\labelenumi{\arabic{enumi}.}
  \setcounter{enumi}{3}
  \tightlist
  \item
    Definir solo e determinar altura dos pontos: Assegura e certifica a
    classificação do solo por diferentes métodos;
  \end{enumerate}
\item
  \begin{enumerate}
  \def\labelenumi{\arabic{enumi}.}
  \setcounter{enumi}{4}
  \tightlist
  \item
    Classificar outros pontos e ruídos: Separar os pontos restantes por
    sua natureza (edificações, corpos d'água etc);
  \end{enumerate}
\item
  \begin{enumerate}
  \def\labelenumi{\arabic{enumi}.}
  \setcounter{enumi}{5}
  \tightlist
  \item
    Quadrículas limpas: Remoção dos ruídos após classificação;
  \end{enumerate}
\item
  \begin{enumerate}
  \def\labelenumi{\arabic{enumi}.}
  \setcounter{enumi}{6}
  \tightlist
  \item
    Produção de mapas DTM (elevação do solo), DSM (elevação do solo +
    vegetação) e DEM (elevação da vegetação);
  \end{enumerate}
\item
  \begin{enumerate}
  \def\labelenumi{\arabic{enumi}.}
  \setcounter{enumi}{7}
  \tightlist
  \item
    Normalização da nuvem: Joga tudo pro mesmo plano;
  \end{enumerate}
\item
  \begin{enumerate}
  \def\labelenumi{\arabic{enumi}.}
  \setcounter{enumi}{8}
  \tightlist
  \item
    Cálculo de métricas LiDAR: O p90 (percentil 90) representa a altura
    em que 90\% dos pontos da nuvem LiDAR se encontram abaixo desta.
    Cada quadrícula possui o seu própio p90 e ele tende a apresentar boa
    correlação com a realidade de campo. A intenção de exisitrem
    diversas métricas LiDAR é a de ver quais delas possuem boa
    correlação com os atributos de interesse.
  \end{enumerate}
\end{itemize}

\begin{figure}

{\centering \includegraphics[width=0.4\linewidth]{IMAGES/als-e-info-campo} 

}

\caption{Esquematização do processo de inventário com ALS e informação terrestre}\label{fig:unnamed-chunk-7}
\end{figure}

\newpage

\subsection{Fluxograma e plots relacionados à Dupla Amostragem proposta
em Inventário
Florestal}\label{fluxograma-e-plots-relacionados-uxe0-dupla-amostragem-proposta-em-inventuxe1rio-florestal}

\begin{figure}

{\centering \includegraphics[width=0.8\linewidth]{IMAGES/fluxogramaDA} 

}

\caption{Fluxograma das etapas de processamento de dados LiDAR para fins de inventário florestal}\label{fig:unnamed-chunk-8}
\end{figure}

\begin{enumerate}
\def\labelenumi{\arabic{enumi}.}
\tightlist
\item
  Carregamento dos pacotes
\end{enumerate}

\begin{enumerate}
\def\labelenumi{\roman{enumi}.}
\tightlist
\item
  Diversos são os pacotes carregados. Os nomes e a utilidade de cada um
  estão descritos na primeira seção do documento. \newpage
\end{enumerate}

\begin{enumerate}
\def\labelenumi{\arabic{enumi}.}
\setcounter{enumi}{1}
\tightlist
\item
  Definição das pastas e local de trabalho
\end{enumerate}

\begin{itemize}
\tightlist
\item
  GitHub

  \begin{enumerate}
  \def\labelenumi{\roman{enumi}.}
  \tightlist
  \item
    C: - pasta raiz
  \item
    GitRepo - diretório em que estão agrupados os arquivos a serem
    upados no GitHub
  \item
    PRJ\_FAZENDAMODELO - pasta do projeto
  \item
    RMD - código e arquivos utilizados na redação do presente documento
  \item
    RESULTADOS - plot da matriz de correlação
  \item
    BATCHR - arquivo tipo R com os scripts utilizados no pré e
    pós-processamento dos dados
  \item
    SAIDASSIG - arquivos gerados no QGIS
  \item
    SHAPES - arquivos de entrada para uso no SIG
  \end{enumerate}
\item
  LiDAR

  \begin{enumerate}
  \def\labelenumi{\roman{enumi}.}
  \tightlist
  \item
    C: - pasta raiz
  \item
    LiDAR - agrupa todos as nuvens de pontos utilizadas no script
  \item
    PRJ\_FAZENDAMODELO - pasta do projeto
  \item
    NUVENS - onde se localizam as nuvens de pontos
  \item
    A13 - reúne as nuvens do ano de 2013
  \item
    TALHOES - nuvem segregada por talhão
  \item
    NoNORM - nuvens com solo classificado
  \item
    SiNORM - nuvens com solo classificado e normalizadas
  \item
    RSTR\_qua - raster apresentando a estimativa da variável de
    interesse para cada talhão
  \end{enumerate}
\end{itemize}

\begin{figure}

{\centering \includegraphics[width=0.8\linewidth]{IMAGES/organizacao-pastas} 

}

\caption{Organização dos diretórios}\label{fig:unnamed-chunk-9}
\end{figure}

\newpage

\begin{enumerate}
\def\labelenumi{\arabic{enumi}.}
\setcounter{enumi}{2}
\tightlist
\item
  Definição das funções para criação dos gráficos
\end{enumerate}

\newpage

\begin{enumerate}
\def\labelenumi{\arabic{enumi}.}
\setcounter{enumi}{3}
\tightlist
\item
  Definição das equações (estudar quais são)
\end{enumerate}

\newpage

\begin{enumerate}
\def\labelenumi{\arabic{enumi}.}
\setcounter{enumi}{4}
\tightlist
\item
  Download e leitura dos dados LiDAR
\end{enumerate}

\begin{enumerate}
\def\labelenumi{\roman{enumi}.}
\tightlist
\item
  Ao todo foram baixadas 6 nuvens de pontos LiDAR, que antes do
  processamento encontravam-se da seguinte maneira:
\end{enumerate}

\begin{figure}

{\centering \includegraphics[width=0.4\linewidth]{IMAGES/pre-clipagem-passo8} 

}

\caption{Nuvens de pontos LiDAR pré-processadas}\label{fig:unnamed-chunk-10}
\end{figure}

\begin{enumerate}
\def\labelenumi{\roman{enumi}.}
\setcounter{enumi}{1}
\tightlist
\item
  As nuvens foram baixadas pelo seguinte link:
  \url{https://github.com/FlorestaR/dados/blob/main/5_LIDARF/Modelo/CLOUDS/}
\end{enumerate}

\newpage

\begin{enumerate}
\def\labelenumi{\arabic{enumi}.}
\setcounter{enumi}{5}
\tightlist
\item
  Download e leitura dos dados em Shapefile\\
\end{enumerate}

\begin{enumerate}
\def\labelenumi{\roman{enumi}.}
\tightlist
\item
  Os shapes foram baixados pelo seguinte link:
  \url{https://github.com/FlorestaR/dados/blob/main/5_LIDARF/Modelo/SHAPES}
\end{enumerate}

\begin{figure}

{\centering \includegraphics[width=0.5\linewidth]{IMAGES/atributosinventariadas} 

}

\caption{Dados contidos nas parcelas inventariadas}\label{fig:unnamed-chunk-11}
\end{figure}

\newpage

\begin{enumerate}
\def\labelenumi{\arabic{enumi}.}
\setcounter{enumi}{6}
\tightlist
\item
  Clipagem da nuvem para eliminação de dados indesejados (mostrar nuvem
  antes e depois)\\
  Antes da clipagem: 68mi pontos\\
  Após clipagem: 18mi pontos
\end{enumerate}

\begin{figure}

{\centering \includegraphics[width=0.4\linewidth]{IMAGES/pre-clipagem-passo8} \includegraphics[width=0.4\linewidth]{IMAGES/pos-clipagem-passo8} 

}

\caption{Comparativo entre as nuvens de pontos antes e após a clipagem}\label{fig:unnamed-chunk-12}
\end{figure}

\newpage

\begin{enumerate}
\def\labelenumi{\arabic{enumi}.}
\setcounter{enumi}{7}
\tightlist
\item
  Classificação
\end{enumerate}

\begin{figure}

{\centering \includegraphics[width=0.4\linewidth]{IMAGES/classificacao-pontos-de-solo} \includegraphics[width=0.4\linewidth]{IMAGES/nuvem-com-pontos-de-solo-classificados-mas-sem-normalizacao} 

}

\caption{Processamento e resultado da classificação de solo das nuvens LiDAR}\label{fig:unnamed-chunk-13}
\end{figure}

\newpage

\begin{enumerate}
\def\labelenumi{\arabic{enumi}.}
\setcounter{enumi}{8}
\tightlist
\item
  Normalização
\end{enumerate}

\begin{enumerate}
\def\labelenumi{\roman{enumi}.}
\tightlist
\item
  A etapa de normalização tem por finalidade nivelar toda a nuvem e é um
  passo que está diretamente correlacionado à classificação do solo.
\end{enumerate}

\begin{figure}

{\centering \includegraphics[width=0.4\linewidth]{IMAGES/nuvensnormalizadas} \includegraphics[width=0.4\linewidth]{IMAGES/nivelamento} 

}

\caption{Resultado da normalização das nuvens}\label{fig:unnamed-chunk-14}
\end{figure}

\newpage

\begin{enumerate}
\def\labelenumi{\arabic{enumi}.}
\setcounter{enumi}{9}
\tightlist
\item
  Recorte da nuvem em tiles 300x300m O recorte da nuvem em tiles menores
  tme a função de facilitar o processamento dos dados, tornando-o mais
  rápido e dinâmico
\end{enumerate}

\begin{figure}

{\centering \includegraphics[width=0.5\linewidth]{IMAGES/Retile} 

}

\caption{Retile da nuvem em quadrados de 300x300m}\label{fig:unnamed-chunk-15}
\end{figure}

\newpage

\begin{enumerate}
\def\labelenumi{\arabic{enumi}.}
\setcounter{enumi}{10}
\tightlist
\item
  Cálculo de métricas para cada voxel (explicar voxel)
\end{enumerate}

\begin{figure}

{\centering \includegraphics[width=0.5\linewidth]{IMAGES/calculo-metricas-voxel} 

}

\caption{Cálculo das métricas para cada voxel}\label{fig:unnamed-chunk-16}
\end{figure}
\newpage

\begin{enumerate}
\def\labelenumi{\arabic{enumi}.}
\setcounter{enumi}{11}
\tightlist
\item
  Seleção de um grupo de métricas para estudo de correlação
\end{enumerate}

\begin{figure}

{\centering \includegraphics[width=0.5\linewidth]{IMAGES/tb-subgrupo-de-metricas-p-analise} 

}

\caption{Tabela com as métricas escolhidas}\label{fig:unnamed-chunk-17}
\end{figure}

\newpage

\begin{enumerate}
\def\labelenumi{\arabic{enumi}.}
\setcounter{enumi}{12}
\tightlist
\item
  Análise da correlação por meio de gráfico (falar do gráfico)
\end{enumerate}

\begin{figure}

{\centering \includegraphics[width=0.5\linewidth]{IMAGES/MatrizDeCorrelacoes} 

}

\caption{Resultado da análise de correlação}\label{fig:unnamed-chunk-18}
\end{figure}

\newpage

\begin{enumerate}
\def\labelenumi{\arabic{enumi}.}
\setcounter{enumi}{13}
\tightlist
\item
  Determinação do modelo de predição
\end{enumerate}

\begin{figure}

{\centering \includegraphics[width=0.5\linewidth]{IMAGES/analise-de-regressao} 

}

\caption{nao sei}\label{fig:unnamed-chunk-19}
\end{figure}

\newpage

\begin{enumerate}
\def\labelenumi{\arabic{enumi}.}
\setcounter{enumi}{14}
\tightlist
\item
  Criação de mapas raster e gráficos
\end{enumerate}

\newpage

\section{Dupla Amostragem}\label{dupla-amostragem}

A dupla amostragem é composta por duas fases (s1 e s2) - LiDAR

\begin{itemize}
\item
  s1: compreende uma gama de variáveis explanatórias para cada ponto
  pertencente a s1. As variáveis explanatórias derivam de informações
  auxiliares disponíveis em grande quantidade ao longo da área
  florestal;\\
\item
  s2: constitui o inventário terrestre feito num número limitado de
  subamostras, onde todas pertencem a s1 e fornecem o valor das
  variáveis de interesse, ex. densidade local.
\item
  É importante que s1 tenha uma grande intensidade amostral, de modo a
  que a média ou o total amostral da medida auxiliar possa ser estimado
  com precisão, sem erros amostrais. É uma variável barata e que pode
  ser encontrada em abundância na floresta. Os estimadores de razão e
  regressão utilizam a relação entre essa variável e a variável de
  interesse (cara e de difícil medição - VTCC) para garantir melhor
  precisão ao cálculo.
\item
  O processo: busca-se estimar a média populacional da medida de
  interesse (ex. em média, quantos m³ de madeira há por hectare?) e
  obter um indicativo da confiabilidade do cálculo (variância). Para
  aumentar a precisão da estimativa da média (diminuir a variância),
  utiliza-se a variável auxiliar que é de fácil observação e está
  associada à variável de interesse (aqui entra estimador de
  razão/regressão). Este movimento de redução da variância da média da
  medida de interesse ocorre de formas diferentes nas DA de
  \textbf{\emph{fase 2 de subamostra (onde todas as parcelas pertencem à
  fase 1) e fase 2 independente (há sorteio, então nem todas as parcelas
  inventariadas de forma intensiva fazem parte da fase 1)}} TA CERTO
  ISSO??? APARENTEMENTE OS DOIS TIPOS DE FASE 2 SE DIVIDEM EM: SORTEIO
  ALEATORIO DAS AMOSTRAS E ESTRATIFICACAO
\item
  Exemplo prático e definições
  (\url{https://www.ipef.br/publicacoes/scientia/nr108/cap09.pdf});
\item
  O erro amostral (E\%) é calculado a partir do desvio padrão S², que
  representa a variação de uma série de médias retiradas da população;
\item
  A alta correlação entre métricas LiDAR e parâmetros biofísicos da
  floresta justificam a adoção da tecnologia na primeira fase da DA,
  além de reduzir a intensidade amostra (custo e trabalho). A amostragem
  dupla ganha maior precisão pela adoção dos estimadores de razão e
  regressão e para isso a média ou o total populacional da variável
  auxiliar deve ser conhecido e sem erro amostral. ;
\end{itemize}

\begin{figure}

{\centering \includegraphics[width=0.6\linewidth]{IMAGES/eq-estimador-regressao} 

}

\caption{Modelo para cálculo da VTCC}\label{fig:unnamed-chunk-20}
\end{figure}

\newpage

\section{Fluxograma e etapas Tripla
amostragem}\label{fluxograma-e-etapas-tripla-amostragem}

Composta por 3 fases, s0, s1 e s2:

\begin{itemize}
\item
  O princípio básico é o de que as variáveis explanatórias derivadas das
  informações auxiliares estão disponíveis em duas frequências
  diferentes. A fase s0 fornece informações sobre toda a área, enquanto
  s1 possui dados adicionais de amostras de s0. Logo, a partir da
  informação terrestre coletada de um número x de parcelas de campo (1
  camada de informação) é possível aferir sobre informações adicionais
  para outras parcelas a partir do uso de preditores (2 camadas de
  informação) \textbf{\emph{É ISSO MESMO? A CAMADA S1 É COMPOSTA POR
  DADOS ESTIMADOS?}} e, por fim, o LiDAR coleta dados sobre todas as
  parcelas (3 camadas de informação).
\item
  Logo, a motivação por trás da TA é a de que a gama de informações de
  s1 adiciona alto poder preditivo às variáveis disponíveis para todas
  as parcelas da área (s0)
\end{itemize}

\begin{figure}

{\centering \includegraphics[width=0.5\linewidth]{IMAGES/esquematizacao-TA} 

}

\caption{Esquema TA com Lidar}\label{fig:unnamed-chunk-21}
\end{figure}

\textbf{Estimativas de pequenas áreas (small area estimation)}

\begin{itemize}
\item
  Para sub áreas da floresta onde há pouca informação terrestre, como em
  G na figura acima. Para essas áreas, o uso da amostragem multifásica
  pode ser mais eficiente, já que utilizam um número reduzido de
  parcelas em campo para chegar à mesma precisão da ACE e ACS. Por outro
  lado, a sub área em questão pode ser pequena demais para justificar a
  adoção de um modelo de regressão separado, uma vez que este pode
  resultar em um intervalo de confiança indesejadamente abrangente;
\item
  A ideia, então, é a de utilizar toda a riqueza de informações presente
  nas amostras de S2 para ajustar o modelo (equação utilizada para o
  cálculo de uma variável de interesse) e aplicá-lo para a sub área em
  questão.
\item
  O potencial viés que surge da aplicação do modelo na sub área é
  corrigido pelo uso de modelos residuais empíricos derivados da área
  amostrada em campo. (não entendi direito essa parte) - trecho no
  texto: The potential bias of applying that model in G is then
  corrected for by using the empirical model residuals derived from that
  small area.
\item
  Caso não existam parcelas de campo na sub área, então deve-se aceitar
  o viés na estimativa e no modelo. São essas as estimativas sintéticas,
  mas apesar do viés é possível calcular a sua variação.
\end{itemize}

\textbf{DESIGN-BASED VS MODEL-DEPENDENT APPROACH (ver dps)}

\begin{itemize}
\item
  Model-dependent approach: as parcelas amostrais são fixas e as
  observações retiradas desses locais são assumidas como variáveis
  aleatórias, assim como a floresta assume o papel de ser o meio
  realizador desse processo estocástico (?????? o que isso significa?).
  Embora os locais das parcelas possam ser arbitrariamente escolhidos, o
  modelo deve escrever adequadamente o processo estocástico, a fim de
  garantir resultados parciais. (????)
\item
  Os estimadores do forestinventory se baseiam na design-based approach.
  Design-based approach baseia-se na randomização dos locais de amostra
  de forma uniforme e independente. Logo, a floresta por si só e
  qualquer valor de densidade local de x (pertencente a) F (floresta)
  são fixos e não um resultado de um processo estocástico (estudar).
  (???)
\item
  Na estrutura do Model-dependent approach as parcelas são fixas e
  resultam em um processo estocástico GPT: what is the difference
  between design-based approach and model-dependent approach?
\end{itemize}

\section{Regressão linear simples}\label{regressuxe3o-linear-simples}

Modelar a relação entre duas variáveis e se expressa na forma de uma
reta no plano cartesiano.

\begin{figure}

{\centering \includegraphics[width=0.8\linewidth]{IMAGES/eq-base-regressao-linear} 

}

\caption{Equação base da regressão linear simples}\label{fig:unnamed-chunk-22}
\end{figure}

\begin{itemize}
\tightlist
\item
  \emph{Y = variável resposta ou variável dependente}\\
\item
  \emph{X = variável independente, regressora ou explicativa}\\
\item
  \emph{Épsilon = erro. Pode ser que a Y seja explicada por X, mas falte
  algo que explique essa variabilidade. É a flutuação aleatória que
  ocorre ao tentar explicar Y por X. Seja por imperfeição do modelo,
  erros na mensuração ou outras variáveis fora de controle.}
\end{itemize}

\newpage

\section{Arborimetria preditiva}\label{arborimetria-preditiva}

\begin{itemize}
\item
  Em florestas plantadas há grande relação entre o DAP e a altura da
  árvore. O procedimento de medição de campo é o seguinte: 1 - DAP de
  todas as árvores da parcela; 2 - altura total de uma amostra da
  parcela; 3 - estima-se a relação entre o DAP e a altura total da
  amostra; 4 - utiliza essa relação para predizer a altura de todas as
  árvores da parcela.
\item
  A seleção das árvores da amostra pode ser feita de forma aleatória ou
  sistemática. A seleção aleatória garante maior confiabilidade,
  enquanto a sistemática é mais simplificada (ex. 1 árvore a cada 15).
\item
  A *\textbf{relação hipsométrica} é a relação média entre o DAP e a
  altura das árvores individuais. A curva sigmóide é o melhor modelo
  para a relação média entre o DAP e a altura.
\end{itemize}

\begin{figure}

{\centering \includegraphics[width=0.45\linewidth]{IMAGES/modelo-nao-linear} 

}

\caption{Quadro de modelos não lineares (esquerda) e lineares (direita)}\label{fig:unnamed-chunk-23-1}
\end{figure}
\begin{figure}

{\centering \includegraphics[width=0.45\linewidth]{IMAGES/modelo-linear} 

}

\caption{Quadro de modelos não lineares (esquerda) e lineares (direita)}\label{fig:unnamed-chunk-23-2}
\end{figure}

\begin{itemize}
\item
  Um modelo é linear quando cada termo é uma constante ou o produto de
  um parâmetro e uma variável preditora. Uma equação linear é construída
  adicionando os resultados para cada termo. Modelos lineares seguem
  esse padrão: Resposta = constante + parâmetro * preditor + \ldots{} +
  parâmetro * preditor. Caso contrário a equação é não-linear.
\item
  O postulado de Cotta define que o volume de uma árvore depende do seu
  diâmetro, altura e forma. Todas as árvores de mesmo diâmetro, altura e
  forma possuem o mesmo volume. Isso serve para o lenho de tronco único.
  O fator forma pode ser visto como um índice quantitativo que
  representa a razão entre o volume sólido e cilíndrico de uma árvore. O
  volume sólido de uma amostra da parcela pode ser medido por cubagem
  rigorosa. O fator forma é igual ao somatório dos volumes sólidos
  dividido pelo somatório dos volumes cilíndricos ({[}pi/4{]} * d² * h).
\item
  O volume de uma dada árvore é o resultado da multiplicação entre o
  fator de forma e o volume cilíndrico do indivíduo. A seleção das
  árvores deve ser aleatória, sem preocupação em selecionar árvores de
  todas as classes de DAP e altura. \textbf{\emph{Vamos usar o LiDAR TLS
  para estimar o diâmetro e a forma?}}
\item
  O volume pode ser calculado em função de diferentes variáveis
  preditoras, como a altura, DAP, altura até a primeira ramificação ou
  outro. Equações de dupla entrada ou equações padrão são aquelas que
  utilizam o DAP e a altura para cálculo do volume.
\item
  A relação entre o volume sólido e o volume cilíndrico tende à uma
  relação de proporcionalidade (a curva que os representa passa pela
  origem). Com a adoção do volume comercial, são consideradas as árvores
  a partir de um tal diâmetro, o que afasta a curva da origem e,
  portanto, não evidencia uma relação proporcional.
\item
  É possível modelar o fator de forma como uma função do DAP ao invés de
  encontrá-lo a partir da relação entre o DAP e a altura. É adequado
  para quando a forma da árvore muda de acordo com o DAP.
\end{itemize}

\begin{figure}

{\centering \includegraphics[width=0.8\linewidth]{IMAGES/dupla-entrada-volume} 

}

\caption{Equação base da regressão linear simples}\label{fig:unnamed-chunk-24}
\end{figure}

\newpage

\section{Silvimetria com medidas
auxiliares}\label{silvimetria-com-medidas-auxiliares}

\begin{itemize}
\item
  Medidas de fácil obtenção: densidade de estande e área basal e possuem
  boa relação com as estimativas de produção (volume). Existem dois
  tipos de relação entre as medidas auxiliares e as de interesse e
  resultam em dois estimadores diferentes: estimadores de razão (razão
  entre a medida de interesse e a auxiliar) e o estimador de regressão
  (relação entre as duas medidas é uma relação linear simples).
\item
  Estimador de razão é a razão entre a medida de interesse e a medida
  auxiliar. A razão pode ser em relação ao total ou às médias das
  medidas. Ex: saber o total das árvores em vias públicas de uma cidade
  (caso 1); saber a proporção das vias públicas sombreadas por árvores
  (caso 2). O primeiro caso se trata de um total - número de árvores na
  cidade -, o segundo caso envolve a razão entre a área sombreada e área
  total das vias públicas. O estimador de razão é uma razão das médias,
  e não uma média das razões.
\end{itemize}

\(R = \frac{ (1/n)\sum^{n}_{i=1} Yi}{ (1/n)\sum^{n}_{i=1} Xi} = \frac{\sum^{n}_{i=1} Yi}{\sum^{n}_{i=1} Xi} = \frac{Total Y}{Total X}\)

\begin{itemize}
\item
  \emph{R} é o estimador de razão
\item
  \emph{i} indica os arvoredos da amostra \emph{(i = 1, 2, \ldots, n)}
\item
  \emph{Yi} é a observação da medida de interesse no arvoredo \emph{i}
\item
  \emph{Xi} é a observação da medida auxiliar no mesmo arvoredo \emph{i}
\item
  O estimador de razão nada mais é do que a soma das médias da medida de
  interesse divida pela soma soma das médias da medida auxiliar. é um
  meio para obter a estimativa da média da variável de interesse com
  maior precisão. Quanto maior a relação entre a medida de interesse e a
  medida auxiliar, menor é a soma de quadrados da expressão
  \(\sum(Yi - R * Xi)^2\) \textbf{\emph{FORMULA VARIANCIA ESTIMADOR DE
  RAZAO E FORMULA DA VARIANCIA POPULACIONAL - PG 296}}
\item
  O estimador de razão requer o conhecimento da \emph{média} ou do
  \emph{total populacional} da variável auxiliar, que é um atributo dos
  arvoredos, a exemplo da densidade de estande e da área basal, o que
  poderia ser obtido a partir de um censo, descartando erros, mas é
  caro. O estimador de razão é utilizado nas situações em que a medida
  de interesse (Y) é diretamente proporcional à medida auxiliar (X).
  Logo, deve satisfazer 3 condições:\\
\end{itemize}

\begin{enumerate}
\def\labelenumi{\arabic{enumi}.}
\tightlist
\item
  A relação média entre Y e X é linear: pode ser representada por uma
  reta;
\item
  A reta X-Y passa pela origem do plano cartesiano;
\item
  O grau de dispersão dos valores de Y (denota a variância) ao redor da
  reta é proporcional aos valores de X.
\end{enumerate}

\begin{itemize}
\tightlist
\item
  O \textbf{Estimador de regressão} é utilizado na existência de uma
  relação linear entre a variável de interesse e a auxiliar, porém a
  reta que as representa não passa pela origem. Logo, o estimador de
  regressão possui uma abordagem mais generalista, pois é eficiente
  mesmo quando as medidas da variável auxiliar e de interesse estão
  negativamente associadas - inversamente proporcionais.
\end{itemize}

\section{R utilizando database
grisons}\label{r-utilizando-database-grisons}

\begin{itemize}
\tightlist
\item
  Para essa etapa será utilzado o database ``grisons''. a fase s1 é
  composta por 306 amostras (phase\_id\_2p) escaneadas com LiDAR e s2
  possui 67 subamostras de s1, de onde se calculou o volume com os dados
  terrestres.
\item
  Para estimadores globais de dupla amostragem, devemos especificar:\\
\end{itemize}

\begin{enumerate}
\def\labelenumi{\arabic{enumi}.}
\tightlist
\item
  O modelo de regressão (formula);
\item
  O dataframe que possui as informações de inventário;
\item
  O objeto de lista (list) \emph{phase-id} contendo: o argumento
  \emph{phase.col} identificando o nome da coluna e especificando para
  s1 ou s2.
\item
  Argumento terrgrid.id, que especificando qual valor numérico indica s2
  na coluna.
\item
  Opcional: coluna contendo os tamanhos de borda
\end{enumerate}

\begin{Shaded}
\begin{Highlighting}[]
\FunctionTok{head}\NormalTok{(grisons)}
\end{Highlighting}
\end{Shaded}

\begin{verbatim}
##   phase_id_2p phase_id_3p boundary_weights      mean    stddev      max
## 2           1           0        0.8133455 10.263023  7.278974 33.17151
## 3           1           0        1.0000000 23.351052 11.372732 45.22998
## 4           1           1        0.8472245  8.259801  5.958564 21.26001
## 6           1           0        1.0000000 21.563423  7.493390 32.65515
## 7           1           0        1.0000000 15.751000  8.447083 33.10999
## 8           1           1        1.0000000 21.160314 13.129320 42.31677
##        q75 smallarea tvol tvol.3p
## 2 15.10999         A   NA      NA
## 3 31.14239         A   NA      NA
## 4 13.06995         A   NA      NA
## 6 27.81149         A   NA      NA
## 7 22.83515         A   NA      NA
## 8 32.40747         A   NA      NA
\end{verbatim}

\begin{itemize}
\tightlist
\item
  Cria-se um modelo para o cálculo do volume a partir da regressão
  múltipla da média das alturas (mean), desvio padrão dessa média
  (stddev), altura máxima (max) e p75 (q75). O modelo retorna a
  estimativa, a variância externa (variância causada por fatores
  externos e que não são contemplados pelo modelo - ex. ansiedade do
  participante ao realizar um teste cognitivo), a variância peso-g
  (contempla a força de um fator geral para a avaliação sobre os itens
  medidos. É como se houvessem pesos diferentes para os critérios -
  lembra uma média ponderada), a quantidade de amostras utilizadas em s1
  e s2 e o R² (correlação entre as variáveis). A variância peso-g
  geralmente é mais preferível.
\end{itemize}

\begin{Shaded}
\begin{Highlighting}[]
\NormalTok{reg2p\_nex }\OtherTok{\textless{}{-}} \FunctionTok{twophase}\NormalTok{(}\AttributeTok{formula =}\NormalTok{ tvol }\SpecialCharTok{\textasciitilde{}}\NormalTok{ mean }\SpecialCharTok{+}\NormalTok{ stddev }\SpecialCharTok{+}\NormalTok{ max }\SpecialCharTok{+}\NormalTok{ q75, }
                      \AttributeTok{data =}\NormalTok{ grisons, }\AttributeTok{phase\_id =} \FunctionTok{list}\NormalTok{(}\AttributeTok{phase.col =} \StringTok{"phase\_id\_2p"}\NormalTok{,}
                      \AttributeTok{terrgrid.id =} \DecValTok{2}\NormalTok{), }\AttributeTok{boundary\_weights =} \StringTok{"boundary\_weights"}\NormalTok{)}
\FunctionTok{summary}\NormalTok{(reg2p\_nex)}
\end{Highlighting}
\end{Shaded}

\begin{verbatim}
## 
## Two-Phase global estimation
##  
## Call: 
## twophase(formula = tvol ~ mean + stddev + max + q75, data = grisons, 
##     phase_id = list(phase.col = "phase_id_2p", terrgrid.id = 2), 
##     boundary_weights = "boundary_weights")
## 
## Method used:
## Non-exhaustive global estimator
##  
## Regression Model:
## tvol ~ mean + stddev + max + q75
## 
## Estimation results:
##  estimate ext_variance g_variance  n1 n2 r.squared
##  383.5354      279.954   271.5057 306 67 0.6428771
## 
## 'boundary_weight'- option was used to calculate weighted means of auxiliary variables
\end{verbatim}

\begin{itemize}
\tightlist
\item
  O estimador exaustivo pode ser aplicado passando adicionalmente um
  vetor que contenha as exatas médias das variáveis explanatórias
  através do argumento \emph{exhaustive}. Este vetor deve ser calculado
  previamente de tal maneira que qualquer ajuste de borda desejado já
  tenha sido aplicado.
\end{itemize}

\begin{Shaded}
\begin{Highlighting}[]
\FunctionTok{colnames}\NormalTok{(}\FunctionTok{lm}\NormalTok{(}\AttributeTok{formula =}\NormalTok{ tvol }\SpecialCharTok{\textasciitilde{}}\NormalTok{ mean }\SpecialCharTok{+}\NormalTok{ stddev }\SpecialCharTok{+}\NormalTok{ max }\SpecialCharTok{+}\NormalTok{ q75, }\AttributeTok{data =}\NormalTok{ grisons, }\AttributeTok{x =} \ConstantTok{TRUE}\NormalTok{)}\SpecialCharTok{$}\NormalTok{x)}
\end{Highlighting}
\end{Shaded}

\begin{verbatim}
## [1] "(Intercept)" "mean"        "stddev"      "max"         "q75"
\end{verbatim}

\textbf{\emph{Não entendi o código acima}}

\begin{itemize}
\tightlist
\item
  O estimador exaustivo pode ser aplicado após definir o vetor com as
  exatas médias, \emph{true.means.Z}:
\end{itemize}

\begin{Shaded}
\begin{Highlighting}[]
\NormalTok{true.means.Z }\OtherTok{\textless{}{-}} \FunctionTok{c}\NormalTok{(}\DecValTok{1}\NormalTok{, }\FloatTok{11.39}\NormalTok{, }\FloatTok{8.84}\NormalTok{, }\FloatTok{32.68}\NormalTok{, }\FloatTok{18.03}\NormalTok{)}
\NormalTok{reg2p\_ex }\OtherTok{\textless{}{-}} \FunctionTok{twophase}\NormalTok{(}\AttributeTok{formula =}\NormalTok{ tvol }\SpecialCharTok{\textasciitilde{}}\NormalTok{ mean }\SpecialCharTok{+}\NormalTok{ stddev }\SpecialCharTok{+}\NormalTok{ max }\SpecialCharTok{+}\NormalTok{ q75,}
                     \AttributeTok{data =}\NormalTok{ grisons, }\AttributeTok{phase\_id =} \FunctionTok{list}\NormalTok{ (}\AttributeTok{phase.col =} \StringTok{"phase\_id\_2p"}\NormalTok{,}
                      \AttributeTok{terrgrid.id =} \DecValTok{2}\NormalTok{), }\AttributeTok{exhaustive =}\NormalTok{ true.means.Z)}

\FunctionTok{summary}\NormalTok{(reg2p\_ex)}
\end{Highlighting}
\end{Shaded}

\begin{verbatim}
## 
## Two-Phase global estimation
##  
## Call: 
## twophase(formula = tvol ~ mean + stddev + max + q75, data = grisons, 
##     phase_id = list(phase.col = "phase_id_2p", terrgrid.id = 2), 
##     exhaustive = true.means.Z)
## 
## Method used:
## Exhaustive global estimator
##  
## Regression Model:
## tvol ~ mean + stddev + max + q75
## 
## Estimation results:
##  estimate ext_variance g_variance  n1 n2 r.squared
##  376.7426     202.5602   187.2787 Inf 67 0.6428771
\end{verbatim}

\end{document}
